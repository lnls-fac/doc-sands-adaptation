\begin{center}
	\textbf{\large Most commonly used symbols.}
\end{center}

\begin{tabularx}{\linewidth}{@{}lX@{}}
	$A$			& Area of a transverse section of a stored beam; or amplitude of an oscillation.\\
%   $A_\text{int}$ & An effective area of colliding beams.\\
	$a$			& Amplitude of an oscillation; invariant part of an amplitude.\\
    $\alpha$	& Time dilation of off-energy orbits (momentum compaction).\\
	$\alpha_i$	& Damping coefficients of oscillations ($1/\tau$); $i = x, z, \epsilon$.\\
    $B$			& Magnetic field strength.\\
	$\beta(s)$	& Betatron functions.\\
	$c$			& Speed of light.\\
	$\gamma$	& Energy in units of rest energy ($E/mc^2$).\\
	$D$			& %Electron number per unit area ($N/A$); or
    				rate of change of radiationloss with energy ($dU/dE$).\\
%	$\delta$	& Half of the angle of intersection of colliding beams.\\
	$E$			& Electron energy.\\
    $E_0$		& Nominal energy of a stored beam.\\
    $\mathscr{E}$ & Electric field strength.\\
    $e$			& Electric charge of an electron.\\
	$\epsilon$	& Energy deviation ($E - E_0$).\\
    $f$			& Frequency of revolution of a synchronous electron ($c/L$).\\
    $G(s)$		& Curvature function of the design orbit ($1/\rho_s$).\\
    $G_0$		& Curvature of the design orbit of an isomagnetic guide field ($G_0 = 1/\rho_0$).\\
	$\zeta(s)$	& Envelope function ($\sqrt{\beta(s)}$).\\
    $\eta(s)$	& Dispersion function.\\
%   $h$			& Beam height.\\
%	$h^*$		& Beam height at the intersection.\\
%	$I$			& Beam current (Nef).\\
	$J_i$		& Partition numbers of the damping rates; $i = x, z, \epsilon$.\\
    $K(s)$		& Focusing functions of the guide field.\\
	$k$			& Rf harmonic number ($\omega_\text{rf}/\omega_\text{r}$); or deviation from nominal value of a focusing function ($\Delta K$).\\
    $L$			& Length of the design orbit ($c/f$).\\
    $\ell$		& Trajectory length.\\
    %; bunch length.\\
    $m$			& Rest mass of an electron.\\
    $N$			& Number of stored electrons in a beam.\\
    %($I/ef$).\\
	$\nu$		& Betatron number.\\
%	$\Delta \nu_0$ & Beam-beam interaction parameter.\\
	$P$			& Radiofrequency power.\\
    $p$			& Electron momentum.\\
%	$Q$			& Quantum excitation strength.\\
	$q$			& Rf overvoltage ($eV/U_0$).\\
%	$R$			& Gross radius of the design orbit ($L/2\pi$).\\
	$r_e$		& Classical electron radius ($e^2/4\pi\epsilon_0 m c^2$).\\
    $\rho_s$	& Radius of curvature of the design orbit at $s$ [$1/G(s)$] .\\
    $\rho_0$	& Magnetic radius in an isomagnetic guide field.\\
	$s$			& Azimuthal (longitudinal) coordinate.\\
    $\sigma$	& Standard deviation of a distribution.\\
	$T(\epsilon)$ & Time for a revolution.\\
	$T_0$		& Revolution period of a synchronous electron ($L/c$).\\
	$t$			& Laboratory time.\\
	$\tau$		& Time displacement between an electron and the center of its bunch ($y/c$).\\
	$\tau_i$	& Damping time constant of an oscillation mode ($i/\alpha_i$); $i = x,z,\epsilon$.\\
	$\varphi(s)$ & Phase variable of betatron oscillations.\\
	$\Phi(\tau)$		&  Pseudopotential energy of the longitudinal oscillations.\\
	$U(\epsilon)$ & Energy loss by radiation in one revolution.\\
	$u$			& Energy of a radiated quantum.\\
	$V(\tau)$	& Effective "Voltage" of the rf system.\\
%	$w$			& Beam width.\\
%	$w^*$		& Beam width at the intersection.\\
	$x$			& Horizontal (radial) displacement from the design orbit.\\
	$y$			& Longitudinal (azimuthal) displacement from a bunch center $(c\tau)$.\\
    $z$			& Vertical (axial) displacement from the design orbit.\\
    $\Omega$	& Angular frequency of the energy (and longitudinal) oscillations.\\
    $\omega_r$	& Angular frequency of revolution ($2\pi/T_0$).\\
	$\omega_\beta$ & Angular frequency of a betatron oscillation ($\nu\omega_r$).
\end{tabularx}
