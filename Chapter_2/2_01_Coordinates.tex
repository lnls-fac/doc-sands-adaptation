\section{Co-ordinates of the motion} \label{sec:2.1}
Electrons are held in a storage ring by the forces from the magnetic guide field. Magnets are disposed along an \underline{ideal orbit} which is generally a smooth, roughly circular or racetrack shaped, closed curve. When the magnet currents are adjusted to any particular set of consistent values the designed fields are intended to be such that an electron of a \underline{nominal energy} $E_0$, once properly started, will move forever along the ideal orbit. All other stored electrons are constrained by the guide field to move in quasi-periodic, stable trajectories in the neighborhood of the ideal orbit. The nature of these stable trajectories is the subject of this part. The treatment will, however, be limited to a so-called linear approximation and will be applied only to electrons of constant energy, ignoring the effects of the radiation loss and the accelerating fields. Such effects will be taken into account later as perturbations of the idealized trajectories. In most rings the ideal orbit lies in a plane, and I shall limit the discussion to such cases; although the extension to the more general cases is relatively straightforward. The presentation will be simplified by presuming that the plane of the ideal orbit lies horizontal.

It is convenient to describe the motion of an electron in terms of coordinates related to the ideal orbit. The instantaneous position of an electron will be specified by $(s, x, y)$ where $s$ is the distance along the ideal orbit from some arbitrary reference point to the point nearest the electron, and $x$ and $y$ are the horizontal and vertical distances from the ideal orbit. See Fig.~\ref{fig:fig7}. We may call $s$ the \underline{azimuthal coordinate}. The horizontal and vertical displacements are, of course, the displacements locally perpendicular to the design orbit. The positive sense of $s$ will be taken in the sense of the electron’s motion, of $x$ in the “outward” direction, and of $y$ in the “upward” direction. I shall often refer to $x$ as the \underline{radial} coordinate.

\begin{figure}[!htb]
	\centering
	\includegraphics[width=0.6\linewidth]{./Figuras/placeholder.png}
	\caption{Coordinates for describing the trajectories.}
	\label{fig:fig7}
\end{figure}

The coordinates $x$ and $y$ will be considered as “small” quantities in the sense that they are assumed to be always much less than the local radius of curvature of the trajectory, and that in considering variations of the magnetic guide field in the vicinity of the ideal orbit, only linear terms in $x$ and $y$ need be retained. These conditions define the linear approximation of our treatment.

Because the design orbit is a closed curve, the azimuthal coordinate $s$ is cyclic. That is, as $s$ increases indefinitely the location in space repeats itself, repeating each time that s increases by the circumference of the orbit. Let’s call this circumference $L$ - and refer to it as the \underline{length} of this design orbit. A physical location on the azimuth may be identified by $s$, or by $s + L$, or by $s + 2L$, and so on. It will from time to time be convenient to use in place of L an equal quantity $2\pi R$, where $R$ is a kind of “effective radius” of the design orbit. It is common -- though strictly improper -- to speak of $R$ as the “mean” radius of the ring.

