\section{Energy loss}\label{sec:4.1}

A relativistic electron which is accelerated in a macroscopic force field will radiate electromagnetic energy at a rate which is proportional to the square of the accelerating
 force. The rate depends on the angle between the force and the electrons velocity and is larger by the factor $\gamma^2 = (E/mc^2)^2$ when the force is perpendicular to the velocity than when the force is parallel to the velocity. In a storage ring the typical longitudinal forces (from the accelerating system) are much smaller than the typical transverse magnetic forces and $\gamma^2$ is a large number indeed, so we need consider only the radiation effects that accompany the magnetic forces.\\
Let $P_\gamma$ stand for the rate of loss of energy by radiation; it may be written\footnote{I shall assume that you are familiar with the classical theory of electromagnetic radiation by relativistic electrons (see e.g., Ref.~\cite{10}) and will only review briefly the results needed for our purposes.}
\begin{align}
	P_\gamma &= \dfrac{2}{3} \dfrac{r_e}{m c} \gamma^2 F_\perp^2 \nonumber \\
    	&= \dfrac{2}{3} \dfrac{r_e c}{(m c^2)^3} E^2 F_\perp^2,
\end{align}
where $m$ is the rest mass of the electron, $r_e = \dfrac{e^2}{m_e c^2}$ is the classical electron radius, and $F_\perp$ is the magnetic force on the electron.

\begin{proof}
We will begin with the expression of the electric field produced by an accelerated electron (considering the velocity still small compared to the velocity of light) at a point of coordinates $(r,\phi,\theta)$, with the electron on the origin; this expression is

\begin{align*}
E = \dfrac{e\dot{v}\sin \theta}{c^2 r}.
\end{align*}
In order to calculate the rate of energy radiated by the electron, we must obtain the expression for the poynting vector $|\mathbf{S}| = (c/4\pi)E^2$ and integrate it over the area $dA$ related to the solid angle $d\Omega$ by $dA = r^2d\Omega = r^2\sin\theta d\theta d\phi$

\begin{align*}
P = \int_{0}^{2\pi}\int_{0}^{\pi}\dfrac{cE^2}{4\pi}r^2\sin\theta d\theta d\phi = \dfrac{2}{3}\dfrac{e^2a^{2}}{c^3}
\end{align*}
where $a$ is the acceleration of the particle. With Newton second law we rewrite the former expression in a convenient manner

\begin{align*}
P = \dfrac{2}{3}\dfrac{e^2}{m^2c^3} \left(\dfrac{d\mathbf{p}}{dt}\cdot\dfrac{d\mathbf{p}}{dt}\right)
\end{align*}

Since we wish to approach the relativistic limit, we must use covariant expressions to generalize our results and the straighforward covariant expression of radiation power is simply

\begin{align*}
P = -\dfrac{2}{3}\dfrac{e^2}{c^3} \dfrac{1}{m^2} \left(\dfrac{d{p_\mu}}{d\tau}\dfrac{dp^\mu}{d\tau}\right),
\end{align*}

where $d\tau = \gamma^{-1}dt$ is the proper time and the Minkowski metric is $g_{\mu \nu} = (+ - - -)$. Performing the scalar product of these quadrivectors we obtain

\begin{align*}
-\dfrac{d{p_\mu}}{d\tau}\dfrac{dp^\mu}{d\tau} = \left(\dfrac{d\mathbf{p}}{d\tau}\right)^2 - \frac{1}{c^2}\left(\dfrac{dE}{d\tau}\right)^2 = \left(\dfrac{d\mathbf{p}}{d\tau}\right)^2 - \beta^2\left(\dfrac{dp}{d\tau}\right)^2
\end{align*}

Consider that there is no transversal momentum (which is the case of an electron on a circular orbit), then $\boldsymbol{{p}} = \boldsymbol{p}_\parallel$ or equivalently $\boldsymbol{\beta} = \boldsymbol{\beta_\parallel}$. Although, there is a longitudinal and a transversal force components, therefore
 there is such components for the derivative of momentum. The parallel acceleration is related to the parallel force by $mc\gamma^3\boldsymbol{\dot{\beta}}_\parallel = \dfrac{d\mathbf{p}_\parallel}{dt} = \mathbf{F}_\parallel$ and the transverse acceleration can be obtained from the Lorentz force by
 $mc \gamma\boldsymbol{\dot{\beta}}_\perp = \dfrac{d\boldsymbol{p}_{\perp}}{dt} = ec\boldsymbol{{\beta}} \times \mathbf{B} = \mathbf{F}_\perp$. Obviously $\mathbf{F}_\parallel \cdot \mathbf{F}_\perp = 0$, thus

 \begin{align*}
 -\dfrac{d{p_\mu}}{d\tau}\dfrac{dp^\mu}{d\tau} = (1- \beta^2)\left(\dfrac{dp_\parallel}{d\tau}\right)^2 + \left(\dfrac{dp_\perp}{d\tau}\right)^2,
 \end{align*}

 using that $1-\beta^2 = \gamma^{-2}$ and $d\tau = \gamma^{-1}dt$, we obtain two components for the radiation power

\begin{align*}
P_\parallel &=   \dfrac{2}{3}\dfrac{e^2}{m^2c^3}F_\parallel^2 \\
P_\perp &=   \dfrac{2}{3}\dfrac{e^2 \gamma^2}{m^2c^3}F_\perp^2 .
\end{align*}

Since we know that $E = \gamma mc^2$, becomes explicit that the radiation power for parallel acceleration does not depends on particle energy, while the transverse radiation power does. That means that applying the same accelerating force on the particle, we will obtain much more radiation power if the acceleration
is transverse to the propagation of the charged particle than if it is longitudinal.
\end{proof}

It will be convenient to define the constant
\begin{align} \label{eq:4.2}
	C_\gamma = \dfrac{4 \pi}{3} \dfrac{r_e}{(m c^2)^3} = 8.85 \times 10^{-5}\, \text{m-GeV}^{-3}.
\end{align}
Then since $F_\perp = ecB$, the radiated power is
\begin{align} \label{eq:4.3}
	P_\gamma = \dfrac{e^2 c^3}{2 \pi} C_\gamma E^2 B^2.
\end{align}
This instantaneous power is proportional to the square of both the energy and the local magnetic field strength. It is sometimes useful to express the magnetic force in terms of the local radius of curvature $\rho$ of the trajectory; then
\begin{align} \label{eq:4.4}
	P_\gamma = \dfrac{c C_\gamma}{2 \pi}\dfrac{E^4}{\rho^2}.
\end{align}
An electron circulating on the design orbit has the nominal energy $E_0$ and moves on the radius $\rho = 1/G$ - see Section~\ref{sec:2.2}. To find the energy $U_0$ radiated in one revolution
 we must integrate $P_\gamma$ with respect to time once around the ring. Since $dt = ds/c$,
\begin{align} \label{eq:4.7}
	U_0 = \dfrac{C_\gamma E_0^4}{2 \pi} \oint G^2(s) ds.
\end{align}
For an isomagnetic guide field\footnote{See Section~\ref{sec:2.2}.} $G = G_0 = 1/\rho_0$ along the curved parts of length $2 \pi \rho_0$ and zero elsewhere, so
\begin{align} \label{eq:4.8}
	U_0 = \dfrac{C_\gamma E_0^4}{\rho_0}.
\end{align}
For a fixed radius $\rho_0$, the energy radiated per turn varies as the fourth power of the electron energy. A 1 GeV electron moving on a 5 meter radius looses 17 keV each revolution.\\
The average power radiated is $U_0/T_0$ where $T_0= L/c$ is the time elapsed during one revolution. For the general guide field
\begin{align} \label{eq:4.9}
	\mean{P_\gamma} = \dfrac{cC_\gamma}{2\pi} E_0^4 \mean{G^2}.
\end{align}
And for an isomagnetic ring,
\begin{align} \label{eq:4.10}
	\mean{P_\gamma} = \dfrac{c C_\gamma E_0^4}{\rho_0 L}, && \text{(isomag.).}
\end{align}
An electron that is not on the ideal orbit radiates at a different rate. Consider first an electron that has the nominal energy $E_0$. but is circulating with a betatron oscillation. It's rate of radiation will be different from an electron moving on the design orbit only because it moves through a slightly different magnetic field - due to its betatron displacement. But at each azimuth its displacement is equally often positive or negative. And we have assumed that the fields have only a linear variation with displacement. So -- to first order in the betatron amplitude the radiated power averaged -- over a betatron cycle is the same as that of an electron on the design orbit.\\
The same is not true of an electron with an energy different from $E_0$. That
case will be analyzed in the next section.
For ultra-relativistic electrons the radiation is emitted primarily along the direction of motion. Most of the radiation is emitted within the angle $1/\gamma$. The radiation reaction force -- and therefore, the accompanying momentum change -- is exactly opposite to the direction
 of motion\footnote{Neglecting quantum effects; see Section~\ref{sec:5.1}}. The only effect of the radiation is then to decrease the energy of the electron without changing its direction
 of motion.
