\section{Radiation Damping Rates}\label{sec:4.4}

Radiation damping effects have now been considered for all three degrees of freedom of an electron in a bunch: the two transverse betatron displacements $x_\beta$ and $y_\beta$ and the energy oscillations -- which show up also in associated oscillations of $\tau$ and $x_\epsilon$. Each of the three oscillation modes has a natural exponential decay with damping coefficients $\alpha_i$ (with $i = x, y$, or $\epsilon$) that can be conveniently expressed as
\begin{align}
	\alpha_i = J_i \alpha_0 = J_i \dfrac{\mean{P_\gamma}}{2 E_0},
\end{align}
with
\begin{align}
	J_x = 1 - \mathscr{D};&& J_y = 1;&& J_\epsilon = 2 + \mathscr{D}.
\end{align}
The damping time constants are just $1/\alpha_i$ so
\begin{align} \label{eq:4.53}
	\tau_i = \dfrac{2 E_0}{J_i \mean{P_\gamma}}.
\end{align}
For an isomagnetic storage ring $\mean{P_\gamma}$ may be taken from Eq.~\eqref{eq:4.10} then
\begin{align} \label{eq:4.54}
	\tau_i = \dfrac{2}{C_\gamma} \dfrac{L \rho_0}{J_i c E_0^3} && \text{(isomag.)},
\end{align}
where $C_\gamma$ is the constant defined in Eq.~\ref{eq:4.2}. In a given storage ring the damping
time constants vary as the inverse cube of the energy. The number $\mathscr{D}$ is a property
 of the guide field and may be evaluated from one of the equations \eqref{eq:2.10}, \eqref{eq:2.12}
 or \eqref{eq:2.13}). The numbers $J_i$ are known as the damping partition numbers since their sum is a constant:
\begin{align} \label{eq:4.55}
	\sum J_i = J_x + J_y + J_\epsilon = 4.
\end{align}
Although I have not actually proved this last result, it does indeed follow from detailed calculations for a general guide field. (See e.g., Ref. \cite{5}.) Such calculations are, however, not really necessary because Robinson has proved on very general grounds a theorem that yields Eq.~\eqref{eq:4.55} directly. The theorem required only that all of the fields acting on the particle are determined a priori and are not in any way influenced by the motion of the electron.
 These conditions apply if we consider only the prescribed magnetic and rf fields of a storage ring.\\
The damping rates for an individual electron -- and more importantly, for the coherent motion of a clump of them -- can be modified from the above numbers if additional forces are introduced
 that depend on the details of the electron motion. Such forces may, for example, come from image currents in the wall of the vacuum chamber or from currents induced by the beam in rf cavities, or from forces from auxiliary electrode systems powered via amplifiers from detectors that sense the displacement of the electrons. In actual rings, the first effect has led to unstable transverse coherent oscillations and the last one has been used to tame them. The
second effect has been both the cause and the cure of unstable longitudinal oscillations of a bunch. Since such effects require the coherent cooperation of many electrons they are beyond the scope of the report and will not be considered further.\\
From Eq.~\eqref{eq:4.55} one would also obtain the more particular result that $J_x+ J_\epsilon = 3$. This result depends, however, on one restrictive assumption -- that the design orbit lies in a plane and that the magnetic fields are symmetric with respect to that plane. We have already referred briefly (at the end of Section~\ref{sec:3.1}) to one of the consequences of dropping this assumption. Off-energy orbits may generally have ``vertical'' displacements $y_\epsilon$ as well as the ``radial'' displacements $x_\epsilon$. Most of the developments
 made in this report become more complicated and, in particular, the partition numbers will not be given by \eqref{eq:4.54}. The ``conservation'' theorem Eq.~\eqref{eq:4.55} will, however,
 remain valid.\\
Two other remarks about the consequences of this theorem are perhaps in order. First, for ``alternating gradient'' guide fields -- such as those used universally in electron synchrotrons
 and in most proton synchrotrons -- the number $\mathscr{D}$ is greater than 1. As a consequence the radial betatron oscillations are antidamped -- and grow exponentially with time at a fixed energy. This effect has not been grave for the synchrotrons because the amplitude growth due to the antidamping is quite small during the acceleration time. It has however, posed a special problem in the adaptation of the CEA synchrotron for use as a storage ring. For this adapta-
tion it has been necessary to install special magnetic devices designed to modify $\mathscr{D}$
 without affecting significantly the other characteristics of the ring.\\
 Finally, you will appreciate that no real guide ever satisfies exactly the postulated symmetry
 of the fields with respect to the plane of the design orbit. The accidental asymmetries
 are generally small but they will, in general, lead to some coupling of the horizontal and vertical betatron oscillations. When such coupling is taken into account, $x$ and $y$ are no longer the coordinates of the normal modes. And the new normal modes will have damping coefficients which are somewhat different from $\alpha_x$ and $\alpha_y$.
