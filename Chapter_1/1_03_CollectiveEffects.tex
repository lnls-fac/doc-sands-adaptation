\section{Collective Effects} \label{sec:1.3}
When the number of electrons in a circulating bunch is large enough (typically, greater than 10$^9$ or so) interactions \emph{among} the electrons of a bunch, or among bunches, become important - and have, in fact, been a serious problem in all electron storage rings. I turn now to a brief listing of the most significant \emph{collective effects}.

-- The \emph{AdA}- or \emph{Touschek-effect}. Two electrons oscillating within a bunch may Coulomb scatter, transferring some of the oscillation energy of each electron from one coordinate to another. The new amplitudes in the second coordinate may lie outside the available aperture, or may contribute to an increase of the bunch dimensions. The Touschek-effect is generally significant only at low energies -- below 1 GeV or so.

-- \emph{Coherent oscillations}. Each electron in a circulating bunch produces \emph{electromagnetic fields} in the vacuum chamber which influence the motion of the other stored electrons. Such \emph{collective} interactions among the electrons can lead to \emph{unstable coherent oscillations}, in which all of the electrons of a bunch oscillate in a collective mode whose amplitude grows exponentially with time. Such coherent oscillations may involve either the transverse or longitudinal motion of the electrons and can lead to a growth of the bunch size or to the loss of electrons from the bunch.

-- Constructive interference of the radiation fields of electrons in a bunch may give rise to \emph{coherent synchrotron radiation}, which can increase the energy loss of individual electrons. (This effect is not believed to be significant in the storage rings now in operation).

To get the high current densities desired in electron storage rings it is generally necessary that the coherent instabilities be suppressed or otherwise controlled. Then the remaining collective effects (which are essentially incoherent) combine with the single particle effects discussed earlier in determining the bunch dimensions. (I am assuming that the strange bunch-lengthening effect observed in many storage rings - which is, as yet, not understood - will ultimately be explained in terms of one or another of the processes already described).

Once one has learned how to make a high current, stored beam, it remains only to prepare two of them and arrange that they collide. Except that, unfortunately, new complications then arise.



