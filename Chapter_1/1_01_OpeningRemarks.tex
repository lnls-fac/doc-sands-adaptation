\section{Opening Remarks}

%%%%%%%%%%%%%%%%%%%%%%%%%%%%%%%%%%%%%%%%%%%%%%%%%%%%%%%%%%%%%%%%%%%%%%%%%%%%%%%%%
%% these opening remarks are out of date. we should write a new one - ximenes. %%
%%%%%%%%%%%%%%%%%%%%%%%%%%%%%%%%%%%%%%%%%%%%%%%%%%%%%%%%%%%%%%%%%%%%%%%%%%%%%%%%%

% Electron storage rings have now come of age. With the successful operation of Adone, experiments will now begin using colliding beams of electrons and positrons with energies of 1 GeV and beyond, expanding the area of storage ring research which was begun at lower energies with the pioneering instruments at Stanford, Frascati, Novosibirsk, and Orsay. Projects under way at Novosibirsk, Cambridge, electrons Hamburg, and Stanford will soon provide stored colliding beams of electrons at even higher energies. Larger numbers of workers will be basing their research in particle physics on these instruments. Many of the physicists who will be using storage rings will not have had a part in their design and construction, and will not initially have a knowledge of their inner workings. The aim of this report is to provide for such physicists a review of basic of the basic physical processes that determine the behavior of electron storage rings -- with a particular concern for their performance as instruments for research in particle physics. Because of this aim, the material is generally not presented in the form which might be most convenient for those who will be interested in the design of storage rings. It is, rather, developed in a form intended to give the using physicist an understanding of the inherent properties of his instrument -- especially those which will have an influence on his observations -- and to give him some feeling for its basic limitations and its ultimate capabilities.

% In the rest of this introduction I give a qualitative description of each of the basic phenomena that play a role in colliding beam storage rings, including a discussion of the factors which determine the luminosity. This first part is intended to provide a background and a vocabulary for the appreciation of the other reports which describe the operating experience with existing rings and the projects for new rings. In the remaining parts I shall consider in detail the theory of those basic single particle processes which determine the ultimate limits on the performance of storage rings. A discussion of the important collective effects, which have lead to many practical difficulties in high intensity rings, is not included here, but will de found in the report of Pellegrini. At the end, I return to a discussion of the limitations on the performance of high energy storage rings, and apply the results to an illustrative example -- the new Stanford design for a 2-3 Gev ring.