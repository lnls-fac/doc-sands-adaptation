\section{Floquet Theorem} \label{sec:FloquetTheorem}

Considering Hill's equation,
\begin{align}\label{eq:Hill}
	x''+K(s)x=0,
\end{align}
where $K(s)$ is a piecewise continuous function in every finite interval and that it is periodic with minimal period $L$. According to Floquet theorem, any solution can be written as the linear combination of the following independent solutions:
\begin{align*}
	x_1(s) &= \zeta(s) e^{i \varphi(s)}, \\
	x_2(s) &= \zeta(s) e^{-i \varphi(s)},
\end{align*}
where the functions $\zeta$ and $\varphi$ satisfy the following periodic boundary conditions for all $s$, and $\nu$ represents a constant phase advance:
\begin{align*}
	\zeta(s) = \zeta(s+L), && \varphi(s+L)-\varphi(s) = 2\pi \nu.
\end{align*}
Plugging this result in equation \eqref{eq:Hill}, it results in a system of differential equations:
\begin{align*}
	\begin{cases}
		2 \zeta' \varphi' + \zeta \varphi'' = 0, \\
		\zeta'' + K(s)\zeta - \zeta \varphi'^2 = 0.
	\end{cases}
\end{align*}
The first differential equation can be written as
\begin{align*}
	\dfrac{d}{ds}( \varphi' \zeta^2 ) &= 0.
\end{align*}
Then, it is easy to show that
\begin{align}
	\varphi'(s) &= \dfrac{C}{\zeta^2(s)},
\end{align}
where $C$ is a constant, which can be set equal to $1$ without loss of generality.
Then, the second equation becomes
\begin{align}
	\zeta'' + K(s) \zeta - \dfrac{1}{\zeta^3}=0.
\end{align}

If the initial conditions $x(0)$, $x'(0)$ are real numbers and $K(s)$ is a real valued function, then a general form of the solution can be written as
\begin{align}
	x(s) = a \zeta(s) \cos(\varphi(s) - \vartheta),
\end{align}
where $a$ and $\vartheta$ are constants determined by the initial conditions.