\section{Basics of Electrons in Magnetic Fields}

An electron with velocity $\bm{v}$ in the present of a magnetic vector field $\bm{B}(\bm{r}) = \bm{B}(\bm{r})$ experiences a Lorentz force, 
\begin{equation}
\bm{F} = - e \; \bm{v} \times \bm{B},
\end{equation}
where $e = 1.6021766208(98)\times10^{-19}\;C$ is the elementary charge fundamental constant. If the magnetic field is colinear with the velocity the force vanishes. Moreover, the Lorentz force for a pure magnetic field does no work on the electron, as can be mathematically derived,
\begin{eqnarray}
\bm{F} \cdot d\bm{r} &=& q \; \left( \bm{v} \times \bm{B} \right) \cdot d\bm{r} \nonumber \\
                     &=& q \; \left( d\bm{r} \times \bm{v} \right) \cdot \bm{B} \\
                     &=& 0 \nonumber,
\end{eqnarray}
since the infinitesimal displacement vector is colinear with the velocity. As a consequence, the speed $v$ and the mechanical energy $E$ of the relativistic electron,
\begin{equation}
E = m c^2 = \gamma m_0 c^2 = \frac{m_0 c^2}{\sqrt{1-(v/c)^2}}
\end{equation}
is conserved. In this case the time derivative of the linear momentum is proportional to the acceleration:
\begin{equation}
\frac{d\bm{p}}{dt} = m \frac{d\bm{v}}{dt} = - e \; \bm{v} \times \bm{B}. 
\end{equation}
This equation describes how the electron velocity evolves with time in the presente of magnetic field. Since the magnitude $v$ of the velocity $\bm{v} = v\bm{\hat{v}}$ is a constant of motion, the above equation can be rewritten for the velocity versor $\bm{\hat{v}}$. Also we can change the independent variable from time $t$ to the trajectory arc length $\ell$ using that $v = d\ell/dt$. The rate at which the direction $\bm{\hat{v}}$ changes with arc length is then given by
\begin{equation}
\frac{d\bm{\hat{v}}}{d\ell} = - \frac{ec}{\beta E} \bm{\hat{v}} \times \bm{B} = -\frac{1}{R_m} \bm{\hat{v}} \times \bm{B},
\end{equation}
where $\beta \equiv v/c$. By defining the magnetic rigidity $R_m$ as
\begin{equation}
R_m(E) = \frac{\beta E}{ec},
\end{equation}
it is possible to finally rewrite the equation for $d\bm{\hat{v}}/d\ell$ as 
\begin{equation}
\frac{d\bm{\hat{v}}}{d\ell} = -\frac{1}{R_m} \bm{\hat{v}} \times \bm{B}.
\end{equation}
This equation can be solved by projecting it on the plane perpendicular to $\bm{B} = B\bm{\hat{y}}$ and expressing the direction of the velocity in terms of two orthogonal vectors $\bm{\hat{z}}$ and $\bm{\hat{x}}$ obeying $\bm{\hat{z}} \times \bm{\hat{x}} = \bm{\hat{y}}$, 
\begin{equation}
\bm{\hat{v}} = \bm{\hat{z}} \cos \theta + \bm{\hat{x}} \sin \theta.
\end{equation} 
The angle $\theta = \theta(s)$ so defined is measured with respect to the $\bm{\hat{z}}$ direction. The velocity in the direction of $\hat{B}$ remains unchanged whereas the two other components of the equation reduce to a single expression for the evolution of the deflected angle of the electron along the trajectory:
\begin{equation}
\frac{d\theta}{d\ell}  = - \frac{B}{R_m}. 
\end{equation} 
This last equation shows that the deflection angle per unit path length is the ratio between the magnetic field intensity and the magnetic rigidity of the electron, which depends solely on its energy. The most useful integral form of the equation reads
\begin{equation}
\label{equ:deflection_angle}
\Delta \theta = - \frac{1}{R_m} \int{B \; d\ell}.
\end{equation} 
Provided the field integral is taken along the actual trajectory, which is not known a priori, the above equation is exact.
A specialized case of what was derived is when the $B$ field is uniform in space. In this situation Eq.(\ref{equ:deflection_angle}) reduces to a simpler form where the deflection angle is proportional to the path length traversed, which implies that the trajectory lies on a circle with bending radius $\rho$ given by
\begin{equation}
|B| \rho = {R_m}.
\end{equation} 

The larger the magnetic rigidity of the electron is - hence the higher its energy is - the more difficult it is for a given magnetic field amplitude to bend the electron trajectory and, therefore, the larger the bending radius is.

For a numerical example, consider the 3 GeV beam at Sirius. Electrons in the beam have, for this nominal energy, a magnetic rigidity of approximately 10 T.m. In order to have deflection of 360 degrees the electrons have to experience an integrated field of $2\pi R_m \approx$ 62.8 T.m. Typical magnetic fields of dipoles are around 1 T, which means that if Sirius were to be build with such uniform field dipoles it would require exactly 62.8 meters of them.