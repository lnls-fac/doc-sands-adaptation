\section{Bunch Length}\label{sec:5.4}

We have just seen that the distribution in the normalized time displacement $\theta$ is as a Gaussian with a standard deviation that is equal to the standard deviation $\sigma_\epsilon$ of the energy oscillations -- refer to Eq.~\eqref{eq:5.50}. It follows that the fluctuating energy oscillations are accompanied by associated fluctuations in the time displacement $\tau$, and that the standard deviation $\sigma_\tau$ of these fluctuations is -- see Eq.~\eqref{eq:5.52} --
\begin{align}
	\sigma_\tau = \dfrac{\alpha}{\Omega E_0} \sigma_\epsilon.
\end{align}
For an isomagnetic guide field Eq.~\eqref{eq:5.45} gives
\begin{align}
	\sigma_\tau^2 = \dfrac{\alpha^2}{\Omega^2} \dfrac{C_q \gamma_0^2}{J_\epsilon \rho_0} && \text{(isomag).}
\end{align}
Taking $\Omega^2$ from Eq.~\eqref{eq:3.43}
\begin{align}
	\sigma_\tau^2 = \dfrac{\alpha T_0 E_0}{e\dot{V}_0} \dfrac{C_q \gamma_0^2}{J_\epsilon \rho_0} = \dfrac{C_q}{(mc^2)^2} \dfrac{\alpha L}{J_\epsilon \rho_0 c} \dfrac{E_0^3}{e \dot{V}_0} && \text{(isomag).}
\end{align}
The spread $\sigma_\tau$ in the time displacement gives when multiplied by $c$, also the spread
of longitudinal displacement from the bunch center -- or, what we may call the bunch half-length.\\
If the energy $E_0$ of the stored beam in a particular storage ring is varied while holding constant the slope of the rf voltage ($\dot{V}_0$), the bunch length will increase with the energy as $E_0^{3/2}$.\\
In several of the storage rings that have been constructed to date the bunch length is observed to be larger than is predicted here by a significant factor which depends on the number of electrons
 in the stored bunch.
